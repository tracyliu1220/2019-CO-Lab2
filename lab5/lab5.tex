\documentclass[12pt, a4paper]{article}

\usepackage{multicol}
\usepackage{geometry}
\usepackage{setspace}
\usepackage{CJKutf8}
\usepackage{amsmath}
\usepackage{listings}

\title{
    \textbf{2019 Spring NCTU Computer Organization} \\
    \large Lab5 Report \\
    \small Tracy Liu
    \author Tracy Liu
    \date{}
}

\geometry{a4paper,left=2cm,right=2cm,top=2.8cm,bottom=3.2cm}
\setlength{\columnsep}{1cm}
\setlength{\baselineskip}{100pt}
\linespread{1.2}

\begin{document}
    \begin{CJK*}{UTF8}{bsmi}
    \begin{center}
        \LARGE\textbf{2019 Spring NCTU Computer Organization} \\
        \large Lab5 Report \\
        \small 0616015 劉姿利、0616092 粘捷 \\
    \end{center}
    \section{Memory Stall Cycles Calculations}
        \begin{center}
            \begin{tabular}{ccr}
            \hline
                Notions & Operations & Delay (cycle) \\
            \hline
                $t_{sa}$ & Send the address & 1 \\
                $t_{al}$ & Access single cache content & 2 \\
                $t_{al1}$ & Access L1 cache content & 1 \\
                $t_{al2}$ & Access L2 cache content & 10 \\
                $t_{am}$ & Access memory content & 100 \\
                $t_{sw}$ & Send a word of data & 1 \\
            \hline
            \end{tabular}
        \end{center}
        \subsection{One-word-wide memory organization}
            $
            \begin{aligned} 
            hit\_cycles  & = t_{sa} + t_{al} + t_{sw} \\ 
                         & = 1 + 2 + 1 = 4 \\
            miss\_cycles & = t_{sa} + 8 \times (t_{sa} + t_{am} + t_{sw} + t_{al}) + t_{al} + t_{sw} \\
                         & = 1 + 8 \times (1 + 100 + 1 + 2) + 2 + 1 = 836 \\
            \end{aligned}  
            $
        \subsection{Wider memory organization}
            $
            \begin{aligned} 
            hit\_cycles & = t_{sa} + t_{al} + t_{sw} \\ 
                       & = 1 + 2 + 1 = 4 \\
            miss\_cycles & = t_{sa} + (t_{sa} + t_{am} + t_{sw} + t_{al}) + t_{al} + t_{sw} \\
                        & = 1 + (1 + 100 + 1 + 2) + 2 + 1 = 108 \\
            \end{aligned} 
            $
        \subsection{Two-level memory organization}
            $
            \begin{aligned} 
            hit\_cycles         & = t_{sa} + t_{al1} + t_{sw} \\ 
                               & = 1 + 1 + 1 = 3 \\
            L1\_miss\_cycles     & = t_{sa} + (t_{sa} + t_{al2} + t_{sw} + t_{al1}) + t_{al1} + t_{sw} \\
                               & = 1 + 4 \times (1 + 10 + 1 + 1) + 1 + 1 = 55 \\
            global\_miss\_cycles & = t_{sa} + 32 \times (t_{sa} + t_{am} + t_{sw} + t_{al2}) + 4 \times (t_{sa} + t_{al2} + t_{sw} + t_{al1}) + t_{al1} + t_{sw} \\
                               & = 1 + 32 \times (1 + 100 + 1 + 10) + 4 \times (1 + 10 + 1 + 1) + 1 + 1 = 3639 \\
            \end{aligned} 
            $
    \newpage
    \section{Differences among Memory Organizations}

    \section{Bonus}
        \subsection{matmul.txt} 
        \begin{lstlisting}[language=C++]
        for (i = 0; i < m; i ++) {
            for (j = 0; j < p; j ++) {
                for (k = 0; k < n; k ++) {
                    load C[i][j];
                    load A[i][k];
                    load B[k][j];
                    do C[i][j] += A[i][k] * B[k][j];
                    store C[i][j];
                }
            }
        }
        \end{lstlisting}

        \subsection{bonus.txt} 
        \begin{lstlisting}[language=C++]
        for (i = 0; i < m; i ++) {
            for (j = 0; j < n; j ++) {
                load A[i][j];
                for (k = 0; k < p; k ++) {
                    load C[i][k];
                    load B[j][k];
                    do C[i][k] += A[i][j] * B[j][k];
                    store C[i][k];
                }
            }
        }
        \end{lstlisting}

    \end{CJK*}
\end{document}
