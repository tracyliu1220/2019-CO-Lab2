\documentclass[12pt, a4paper]{article}

\usepackage{multicol}
\usepackage{geometry}
\usepackage{setspace}
\usepackage{CJKutf8}
\usepackage{amsmath}
\usepackage{listings}
%\usepackage{indentfirst}

\title{
    \textbf{2019 Spring NCTU Computer Organization} \\
    \large Lab5 Report \\
    \small Tracy Liu, Chieh Nien
    \author Tracy Liu, Chieh Nien
    \date{}
}

\geometry{a4paper,left=2cm,right=2cm,top=2.8cm,bottom=3.2cm}
\setlength{\columnsep}{1cm}
\setlength{\baselineskip}{100pt}
\linespread{1.2}
%\setlength{\parindent}{4em}

\begin{document}
    \begin{CJK*}{UTF8}{bsmi}
    \begin{center}
        \LARGE\textbf{2019 Spring NCTU Computer Organization} \\
        \large Lab5 Report \\
        \small 0616015 劉姿利、0616092 粘捷 \\
    \end{center}
    \section{Memory Stall Cycles Calculations}
        \begin{center}
            \begin{tabular}{ccr}
            \hline
                Notions & Operations & Delay (cycle) \\
            \hline
                $t_{sa}$ & Send the address & 1 \\
                $t_{al}$ & Access single cache content & 2 \\
                $t_{al1}$ & Access L1 cache content & 1 \\
                $t_{al2}$ & Access L2 cache content & 10 \\
                $t_{am}$ & Access memory content & 100 \\
                $t_{sw}$ & Send a word of data & 1 \\
            \hline
            \end{tabular}
        \end{center}
        \subsection{One-word-wide memory organization}
            $
            \begin{aligned}
            hit\_cycles  & = t_{sa} + t_{al} + t_{sw} \\
                         & = 1 + 2 + 1 = 4 \\
            miss\_cycles & = t_{sa} + 8 \times (t_{sa} + t_{am} + t_{sw} + t_{al}) + t_{al} + t_{sw} \\
                         & = 1 + 8 \times (1 + 100 + 1 + 2) + 2 + 1 = 836 \\
            \end{aligned}
            $
        \subsection{Wider memory organization}
            $
            \begin{aligned}
            hit\_cycles & = t_{sa} + t_{al} + t_{sw} \\
                       & = 1 + 2 + 1 = 4 \\
            miss\_cycles & = t_{sa} + (t_{sa} + t_{am} + t_{sw} + t_{al}) + t_{al} + t_{sw} \\
                        & = 1 + (1 + 100 + 1 + 2) + 2 + 1 = 108 \\
            \end{aligned}
            $
        \subsection{Two-level memory organization}
            $
            \begin{aligned}
            hit\_cycles         & = t_{sa} + t_{al1} + t_{sw} \\
                               & = 1 + 1 + 1 = 3 \\
            L1\_miss\_cycles     & = t_{sa} + (t_{sa} + t_{al2} + t_{sw} + t_{al1}) + t_{al1} + t_{sw} \\
                               & = 1 + 4 \times (1 + 10 + 1 + 1) + 1 + 1 = 55 \\
            global\_miss\_cycles & = t_{sa} + 32 \times (t_{sa} + t_{am} + t_{sw} + t_{al2}) + 4 \times (t_{sa} + t_{al2} + t_{sw} + t_{al1}) + t_{al1} + t_{sw} \\
                               & = 1 + 32 \times (1 + 100 + 1 + 10) + 4 \times (1 + 10 + 1 + 1) + 1 + 1 = 3639 \\
            \end{aligned}
            $
    \newpage
    \section{Results of Calculations}
        \begin{center}
            \begin{tabular}{crrrr}
            \hline
                Testcases & Exe Cycles & One-word-wide & Wider & 2-level \\
            \hline
                a1xb1 & 1553    & 6016      & 1648     & 12144 \\
                a2xb2 & 6141    & 19072     & 5968     & 32736 \\
                a3xb3 & 183925  & 7626560   & 1068008  & 791552 \\
                a4xb4 & 5781957 & 251426432 & 35098320 & 998152040 \\
            \hline
            \end{tabular}
        \end{center}
    \section{Differences among Memory Organizations}
        \begin{itemize}
            \item Wider memory organization 因為有比較大的 bandwidth 所以可以一次傳輸多個 words 的 data,減少 data transfer 需要的 cycles,而 2 level memory organization 可以讓某些在 L1 或 L2 hit 的 address 以更少的 cycles 傳輸給 processor。
            \item We can see that increasing bandwidth indeed reduce the miss penalty (from 836 to 108). While with using L-2 cache, the L-2 miss penalty is much higher, so we need to focus on reducing miss rate to avoid total memory stall cycle higher then original memory organizations (Fig1.(a)).
        \end{itemize}


    \section{Bonus}
        \subsection{matmul.txt C++ style code}
        \begin{lstlisting}[language=C++]
        for (i = 0; i < m; i ++) {
            for (j = 0; j < p; j ++) {
                for (k = 0; k < n; k ++) {
                    load C[i][j];
                    load A[i][k];
                    load B[k][j];
                    do C[i][j] += A[i][k] * B[k][j];
                    store C[i][j];
                }
            }
        }
        \end{lstlisting}

        \newpage

        \subsection{bonus\_matmul.txt C++ style code}
        \begin{lstlisting}[language=C++]
        for (i = 0; i < m; i ++) {
            for (j = 0; j < n; j ++) {
                load A[i][j];
                for (k = 0; k < p; k ++) {
                    load C[i][k];
                    load B[j][k];
                    do C[i][k] += A[i][j] * B[j][k];
                    store C[i][k];
                }
            }
        }
        \end{lstlisting}

        \subsection{Explanation}
            我們將 load A[i][j] 提到第二層迴圈,這樣就可以將原來要 load A[i][j] $m \times n \times p$ 次減少到 $m \times n$ 次,減少 require data 的次數。
        \subsection{Test and Delay Cycles Calculations for Bonus}
            \textbf{Compiling:} make bonus \\
            \textbf{Execution:} ./simulate\_caches [input\_filename] [output\_filename] \\
            the output format is exactly the same as the original one. \\

        \subsection{Results of Bonus}
            \begin{center}
                \begin{tabular}{crrrr}
                \hline
                    Testcases & Exe Cycles & One-word-wide & Wider & 2-level \\
                \hline
                    a1xb1 & 1313    & 5824     & 1456    & 12000 \\
                    a2xb2 & 4797    & 18176    & 5072    & 32064 \\
                    a3xb3 & 142453  & 1032000  & 215912  & 454912 \\
                    a4xb4 & 4555205 & 31508096 & 6719696 & 40791860 \\
                \hline
                \end{tabular}
            \end{center}

    \end{CJK*}
\end{document}
